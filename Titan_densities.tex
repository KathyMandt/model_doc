The densities are determined by chemistry and diffusion.

\section{First approximation}

We use here the barometric equation~\ref{Titan:Dens_init}
\begin{equation}
\conc(z) = \conc(\zmin) \exp\left(-\frac{z - \zmin}
                                        {\left(\RTitan + z\right)
                                         \left(\RTitan + \zmin\right)
                                         \frac{\Nav\cdot\kb\cdot T(z)}{\Guni\cdot\MTitan\cdot\mean{\Mm}}
                                        }
                            \right)
\label{Titan:Dens_init}
\end{equation}
with \mean{\Mm}\ begin the mean molar mass of the atmosphere.

\begin{remark}
Beware with the units!
\begin{itemize}
\item $z$         in \unit{km},
\item \RTitan\    in \unit{km} in the file \planetConstantsHeader,
\item \Nav\       in \unit{mol^{-1}} in the file \antiochPhysicalConstantsHeader,
\item \kb\        in $\unit{J\,K^{-1}} \equiv \unit{kg\,m^2s^{-2}}$ in file \planetConstantsHeader,
\item $T$\        in \unit{K},
\item \Guni\      in \unit{m^3kg^{-1}s^{-2}} in file \planetConstantsHeader,
\item \MTitan\    in \unit{kg} in file \planetConstantsHeader,
\item \mean{\Mm}\ in \unit{g\,mol^{-1}}.
\end{itemize}
Thus some adaptations are necessary to obtain a unitless coefficient in the exponential,
changing the term $\left(\RTitan + z\right)$ or $\left(\RTitan + \zmin\right)$ in \unit{m}
and the term \mean{\Mm} in \unit{kg\,mol^{-1}}.
\end{remark}
